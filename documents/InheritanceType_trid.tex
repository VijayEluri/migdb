%komentář
\documentclass[11pt,a4paper]{article}   	%základní popis dokumentu (velikost písma, velikost papíru, typ dokumentu)
\usepackage[utf8]{inputenc}              	%nastavení výchozího kódování textu
\usepackage[czech]{babel}                	%nastavení českých znaků
\usepackage[pdftex]{graphicx}				%umožňuje vkládání obrázků v jpg, png
\usepackage{amsfonts}						%package obsahujici symboly mnozin treba - \mathbb{N}
%\usepackage{amssymb}						%stejna funkce jako amsfonts

\begin{document}                            %začátek dokumentu

	\section{ Vysvětlivky v dokumentu: }
		\begin{itemize} 
  			\item *element značí, že daný element je reference
  			\item zápis \begin{math}g_1 - g_3\end{math} znamená guardy indexované od 1 do 3, přijde mi to jako nejjednodušší značení 				
  			\item K [Column] značí kolekci elementů typu Column	
  			\item predek(i) je třída nacházející se o i úrovní výše v stromu dědičnosti, tzn.
				dědí od ní třída, na které je testován guard
			\item $\mathbb{N}$ je zápis množiny přirozených čísel
			\item root - index kořenu hierarchického stromu - nemá předka  
		\end{itemize}		

	%\texttt je v dokumentu kvuli zvyrazneni, ale nejspise je nadbytecny
	\section{Obecné guardy}
		Obecné guardy by měly splňovat všechny třídy, které jsou serializovatelné, jde
		zejména o to, aby nebyly Primitive a nebyly transientní
		\begin{itemize}
			\item 	\texttt	{
						$g_1$ : isPrimitive = false $\wedge$ isTransient = false  
					}
			\item 	\texttt {
						$g_2 : \exists i \in \mathbb{N} $: properties [i] . isID =
						true \&\& properties[i] . isTransient = false
					}
		\end{itemize}
		
 	\section{Třídy bez serializovatelného předka}
 		\begin{itemize}
 			\item 	\texttt {
 						$g_3$ : parent = NULL $\vee$ parent.isAbstract = true $\vee$
 						$\forall i \in \mathbb{N}: i < root $ predek(i).isTransient 
 					}
 		 	\item	$\omega_1 (\pi_1, g_1 - g_3)$
 		 	\item 	$\pi_1 : E \to $ \{table,  K[Column], primaryKey\}
 		 	\item 	mapovací pravidlo namapuje třídu na nově vzniklou tabulku,
 		 			množinu jejich sloupců a primární klíč
 		\end{itemize}
 	\section{Strom dědičnosti s implicitním inheritenceType}
 		\begin{itemize}
 		  	\item 	\texttt {
 		  					$ g_4: inheritanceType = NULL \wedge parent \neq NULL
 		  					\forall j \in \mathbb{N} : j \leq root : predek(j).inheritanceType =
 		  					NULL$ }
 		  	\item 	$\omega_2 (\pi_2, g_1 - g_2 \wedge g_4) $
 		  	\item 	$\pi_2 : E \to$ \{ table,  K[Column], *foreignKey\}
 		  	\item	implicitní typ Joined vytvoří novou tabulku, sloupce a referuje
 		  			přes FK na PK předka
 		\end{itemize}
 		
 	\section{Třídy s jednoduchou dědičností}
 	 	\subsection{Joined}
 			\begin{itemize}
 			  	\item 	\texttt {
 			  				$g_5 : (parent \neq NULL) \wedge (( inheritanceType =
 			  				parentInheritanceType \wedge inheritanceType = Joined) \vee
 			  				inheritanceType = NULL \wedge \exists i \in \mathbb{N}: \forall j \in
 			  				\mathbb{N} : j < i : predek(j).inheritanceType = NULL ,
 			  				predek(i).inheritanceType = Joined )$ 
 			  			}
 				\item 	$\omega_3 (\pi_2, g_1 - g_2 \wedge g_5)$
 				\item	$\pi_2 : E \to \{ table, $ K[Column], *foreignKey\}
				\item	typ Joined vytvoří novou tabulku, sloupce a referuje na PK předka 	 			
 			\end{itemize}
 			
		\subsection{TablePerClass}
			\begin{itemize} 
			  	\item 	\texttt{
			  				$g_6 : (parent \neq NULL) \wedge (( inheritanceType =
 			  				parentInheritanceType \wedge inheritanceType = TablePerClass) \vee
 			  				inheritanceType = NULL \wedge \exists i \in \mathbb{N}: \forall j \in
 			  				\mathbb{N} : j < i : predek(j).inheritanceType = NULL ,
 			  				predek(i).inheritanceType = TablePerClass )$
 			  			}
			  	\item 	$\omega_5 (\pi_4, g_1 - g_2 \wedge g_8) $
			  	\item 	$\pi_4 : E \to \{ table, $ K[Column], primaryKey\}
				
			\end{itemize}
		
		\subsection{SingleTable}
 			\begin{itemize}
 			  \item \texttt{
 			  			$g_7 : (parent \neq NULL) \wedge (( inheritanceType =
 			  				parentInheritanceType \wedge inheritanceType = SingleTable) \vee
 			  				inheritanceType = NULL \wedge \exists i \in \mathbb{N}: \forall j \in
 			  				\mathbb{N} : j < i : predek(j).inheritanceType = NULL ,
 			  				predek(i).inheritanceType = SingleTable )$
 			  		}
 			  \item 	$\omega_4 (\pi_3, g_1 - g_2 \wedge g_7) $
 			  \item		$\pi_3 : E \to \{ *table, $ K[Column],*K[Column] *primaryKey\}
 			  \item 	Typ SingleTable referencuje tabulku, primarni klic, referencuje
 			  			sloupce predku a vytvori nove sloupce vstupni tridy 
 			\end{itemize}	


	\section{Přechodové třídy mezi různými inheritanceTypy}
		\subsection{sth to SingleTable}
			\begin{itemize} 
			  	\item 	\texttt{
				  				$g_9 : parent \neq NULL \;\exists i \in \matbb{N}: \forall j \in
				  				\mathbb{N} j < i : predek(j).inheritanceType = NULL,
				  				(predek(i).inheritanceType = TablePerClass \vee
				  				predek(i).inheritanceType =  Joined), self.inheritanceType = SingleTable$ 
				  		}
			  	\item 	$\omega_6 (\pi_4, g_1 - g_2 \wedge g_9) $
				\item	$\pi_3 : E \to \{ *table, $ K[Column], *primaryKey\}
			\end{itemize}
			
			
			%mapovaci pravidlo a funkce
			

		\subsubsection{sth to Joined}
			\begin{itemize}
			  	\item 	\texttt{
			  					$g_(10) : parent \neq NULL \;\exists i \in \mathbb{N} : \forall j
			  					\in \mathbb{N} j < i : predek(j).inheritanceType = NULL, (
								predek(i).inheritanceType = TablePerClass \vee predek(i).inheritanceType
								= Joined), self.inheritanceType = SingleTable $
			  			}
			  	\item 	$\omega_5 (\pi_4, g_1 - g_2 \wedge g_10) $
			  	\item 	$\pi_4 : E \to \{ table, $ K[Column], FK\}
			\end{itemize}
\end{document}