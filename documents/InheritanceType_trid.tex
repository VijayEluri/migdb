%komentář
\documentclass[11pt,a4paper]{article}   	%základní popis dokumentu (velikost písma, velikost papíru, typ dokumentu)
\usepackage[utf8]{inputenc}              	%nastavení výchozího kódování textu
\usepackage[czech]{babel}                	%nastavení českých znaků
\usepackage[pdftex]{graphicx}				%umožňuje vkládání obrázků v jpg, png
\usepackage{amsfonts}						%package obsahujici symboly mnozin treba - \mathbb{N}
%\usepackage{amssymb}						%stejna funkce jako amsfonts

\begin{document}                            %začátek dokumentu

	\section{ Vysvětlivky v dokumentu: }
		\begin{itemize} 
  			\item *element značí, že daný element je reference
  			\item zápis \begin{math}g_1 - g_3\end{math} znamená guardy indexované od 1 do 3, přijde mi to jako nejjednodušší značení 				
  			\item K [Column] značí kolekci elementů typu Column	
  			\item precedessor(i) je třída nacházející se o i úrovní výše v stromu dědičnosti, tzn.
				dědí od ní třída, na které je testován guard\\ Matematicky vyjádřeno:
				precedessor(1) = parent\\
				precedessor(i) = precedessor(i - 1).parent\\
				$i \in \mathbb{N}$ 
			\item $\mathbb{N}$ je zápis množiny přirozených čísel
			\item root - index kořenu hierarchického stromu - nemá předka  
		\end{itemize}		

	\section {Primitivni typy}		
		
		\begin{itemize}
 			\item	$guard : isPrimitive = true$
 		 	\item	$\omega_1 (\pi_0, guard)$
 		 	\item	$\pi_0 : class \to  {K[column]}$
 		 	\item   $class \in Class, column \in TableColumn$
 		 	\item	Třída, která je primitivní se nenamapuje do databázového modelu na Tabulku.
					Tyto tridy se namapuji na Sloupce v databázi daného typu.
 		\end{itemize}
		
	\section{EmbeddedClass}
		\begin{itemize}
		  \item 	\item	$guard_1 : isPrimitive = true$
 		 	\item	$\omega_1 (\pi_1, guard1)$
 		 	\item	$\pi_1 : class \to  {column}$
 		 	\item   $class \in Class, column \in TableColumn$
 		 	\item	Embedded třídy se namapují na sloupce třídy, na níž jsou embedded
		\end{itemize}	
		
	
	\section{Obecné guardy}
		Obecné guardy by měly splňovat všechny třídy, které jsou serializovatelné, jde
		o to, aby nebyly Primitive a měly právě jedno ID a nebyly embedded
		\begin{itemize}
			\item	$g_1 : isPrimitive = false \wedge isEmbedded = false$
			\item	$g_2 : \exists i \in \mathbb{N} : properties [i] . isID = true$
		\end{itemize}
	
 	\section{Třídy bez serializovatelného předka}
 		\begin{itemize}
 			\item	$g_3 : parent = NULL$ 
 		 	\item	$\omega_1 (\pi_2, g_1 - g_3)$
 		 	\item 	$\pi_2 : class \to  \{table,  K[column], primaryKey\}$
 		 	\item 	$class \in Class, column \in TableColumn, table \in Table,
 		 			primaryKey \in PrimaryKey$
 		 	\item 	mapovací pravidlo namapuje třídu na nově vzniklou tabulku,
 		 			množinu jejich sloupců a primární klíč
 		\end{itemize}
 		 		
 	\section{Třídy s jednoduchou dědičností}
 		Všechny typy dědičnosti předpokládají již namapovanou třídu předka, není
 		nutný předpoklad namapování všech předchů v hierarchickém stromě,
 		tento je rekurzivně splněn, pokud je namapován předek, matematicky
 		znězorněno: \\ 
 		$g_4 : parent.resolve(Table) = true$
 	 	\subsection{Joined nebo implicitní inteheritanceType}
 			\begin{itemize}
 			  	\item 	$ g_5: inheritanceType = \_\_not\_defined" \wedge parent \neq
 			  			NULL \forall j \in \mathbb{N} : j \leq root :
 			  			precedessor(j).inheritanceType = "\_\_not\_defined"$
 			  	\item	$g_6 : (parent \neq NULL) \wedge (( inheritanceType =
 			  			parent.InheritanceType \wedge inheritanceType = Joined) \vee
 			  			inheritanceType = NULL \wedge \exists i \in \mathbb{N}: \forall j \in
 			  			\mathbb{N} : j < i : precedessor(j).inheritanceType = NULL ,
 			  			precedessor(i).inheritanceType = Joined ) $ 
 				\item 	$\omega_3 (\pi_3, g_1 \wedge g_2 \wedge g_4 \wedge ( g_5 \vee g_6))$
 				\item	$\pi_3 : class \to \{ table,  K[column], * parentPrimaryKeyColumn,
 						K[precedessorColumn], foreignKey, primaryKey\}$
 				\item 	$class \in Class, column \in TableColumn, table \in Table,
 		 				primaryKey \in PrimaryKey, foreignKey in ForeignKey,
 		 				parentPrimaryKeyColumn, parentColumn \in TableColumn$ \\
				\item	typ Joined vytvoří novou tabulku, sloupce a referuje na id sloupec
						předka
				\item	data se budou vkládat do všech sloupců tabulek předků označované jako
						$K[precedessorColumn]$
 			\end{itemize}
 			
		\subsection{TablePerClass}
			\begin{itemize} 
			  	\item	$g_7 : (parent \neq NULL) \wedge (( inheritanceType =
 			  			parent.InheritanceType \wedge inheritanceType = TablePerClass) \vee
 			  			inheritanceType = NULL \wedge \exists i \in \mathbb{N}: \forall j \in
 			  			\mathbb{N} : j < i : precedessor(j).inheritanceType = NULL ,
 			  			precedessor(i).inheritanceType = TablePerClass )$
			  	\item 	$\omega_5 (\pi_4, g_1 \wedge g_2 \wedge g_4 \wedge g_7) $
			  	\item 	$\pi_4 : class \to \{ table,  K[Column], primaryKey\}$
			  	\item 	$class \in Class, column \in TableColumn, table \in Table,
 		 				primaryKey \in PrimaryKey$
			\end{itemize}
		
		\subsection{SingleTable}
 			\begin{itemize}
 			  \item		$g_8 : (parent \neq NULL) \wedge (( inheritanceType =
 			  			parentInheritanceType \wedge inheritanceType = SingleTable) \vee
 			  			inheritanceType = NULL \wedge \exists i \in \mathbb{N}: \forall j \in
 			  			\mathbb{N} : j < i : precedessor(j).inheritanceType = NULL ,
 			  			precedessor(i).inheritanceType = SingleTable )$
 			  \item 	$\omega_4 (\pi_3, g_1 \wedge g_2 \wedge g_4 \wedge g_8) $
 			  \item		$\pi_3 : class \to \{ *table,  K[column],*K[treeColumn]
 			  			*primaryKey\}$
 			  \item 	$class \in Class, column \in TableColumn, table \in Table,
 		 				primaryKey \in PrimaryKey, treeColumn \in TableColumn$
 			  \item 	Typ SingleTable referencuje tabulku, primární klíč,
 			  			vytvoří nové sloupce vstupní třídy, součástí namapované tabulky jsou
 			  			sloupce celé hierarchie 
 			  \item		Na \tetxbf{datové} vrstě budou data vkládána i do sloupců
 			  			hierarchické struktury v zápisu označovaných jako K[treeColumn]
 			\end{itemize}	


	\section{Přechodové třídy mezi různými inheritanceTypy - nedodělané}	
		\subsection{sth to SingleTable}
			\begin{itemize} 
			  	\item	$g_9 : parent \neq NULL \;\exists i \in \matbb{N}: \forall j \in
						\mathbb{N} j < i : precedessor(j).inheritanceType = NULL,
				  		(precedessor(i).inheritanceType = TablePerClass \vee
				  		precedessor(i).inheritanceType =  Joined), self.inheritanceType = SingleTable$ 
			  	\item 	$\omega_6 (\pi_4, g_1 - g_2 \wedge g_9) $
				\item	$\pi_3 : E \to \{ *table,  K[Column], *primaryKey\}$
			\end{itemize}
			
			
			%mapovaci pravidlo a funkce
			

		\subsubsection{sth to Joined}
			\begin{itemize}
			  	\item	$g_(10) : parent \neq NULL \;\exists i \in \mathbb{N} : \forall j
			  			\in \mathbb{N} j < i : precedessor(j).inheritanceType = NULL, (
						precedessor(i).inheritanceType = TablePerClass \vee precedessor(i).inheritanceType
						= Joined), self.inheritanceType = SingleTable $
			  	\item 	$\omega_5 (\pi_4, g_1 - g_2 \wedge g_10) $
			  	\item 	$\pi_4 : E \to \{ table, K[Column], FK\}$
			\end{itemize}
\end{document}