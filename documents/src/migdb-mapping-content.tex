\chapter{Mapování tříd do db}
\label{chap:mmmapping}

\newcommand{\ps}{\mathbb{P}}

Tento dokument zavádí pojmy a značení pro popis mapování JAM
metamodelu do databázového metamodelu. Zajímá nás zde zatím jen
transformace business modelu na databázové schéma, později zřejmě
přidáme ještě další vlastnosti. V první chvíli upustíme od zcela
dokonalého popisu (hlavně co se vstupních a výstupních objektů týče),
ale i tak nám zavedené pojmy jistě zjednoduší domluvu.

$E$ ... Množina všech možných objektů vstupního modelu (intance tříd
JAM metamodelu). Obsahuje hlavně třídy a jejich property. Libovolnou
podmnožinu $E$ nazýváme \emph{modelem}.

$D$ ... Množina všech možných objektů výstupního modelu (instance tříd
z metamodelu popisující db schéma). Obsahuje tabulky, sloupce, klíče,
atd. Libovolnou podmnožinu $D$ nazýváme \emph{schématem}.

$\ps(X)$ ... potenční množina množiny X (power set, množina všech
podmnožin).

Funkce tvaru $\pi : E \to \ps(D)$ nazýváme \emph{mapovacími
  funkcemi}. Mapovací funkce zobrazuje element vstupního modelu na
množinu elementů výstupního modelu a je základním prvkem při
konstrukci složitějších transformací.  Množinu všech takových funkcí
značíme $\Pi = E \to \ps(D)$.

Uspořádanou dvojici $\omega = [\pi, g] \in \Pi \times (E \to \{0,
1\})$ nazýváme \emph{mapovacím pravidlem}, funkci $g$ pak
\emph{guardem} mapovacího pravidla. Guard má pro mapovací pravidlo
význam podmínky, která musí být splněna, aby bylo pravidlo aplikováno,
tedy aby byla užita mapovací funkce $\pi$. Lépe patrné je to z
definice obrazu elementů níže. Množinu všech mapovacích pravidel
značíme $\Omega = \Pi \times (E \to \{0, 1\})$.

\emph{Obraz} elementu $e \in E$ pro pravidlo $\omega = [\pi, g] \in
\Omega$ definujeme jako 
\begin{equation}
  db(e,\omega) : E \times \Omega \to \ps(D) = \left\{
  \begin{array}{rl}
    \varnothing & \text{pokud } g(e) = 0,\\
    \pi(e) & \text{pokud } g(e) = 1.\\
  \end{array} \right.
\end{equation}


\emph{Smyslem mapovacích funkcí a pravidel je možnost na elementární
  úrovni popsat jaké databázové objekty jsou třeba pro serializaci
  nějakého elementu ze vstupního modelu. Různé mapovací funkce mohou
  vracet různé výstupní objekty a realizovat tak různá dílčí
  mapování. Často budeme chtít, aby výsledkem funkce byla prázdná
  množina (například ve smyslu, že daná funkce neumí mapovat nějakém
  vstupní objektu). Nebo aby byla funkce použita jen v situaci, kdy
  jsou splněné dodatečné podmínky. Např. třída \emph{Person} potřebuje
  k standardní serializaci tabulku, a minimálně sloupec pro primární
  klíč. Výsledkem jednoduché mapovací funkce jsou tedy tyto dva
  objekty. V případě, že se ale jedná o embedded třídu, chceme, aby se
  tato třída zobrazila na prázdnou množinu (nemá být serializována
  sama o sobě, ale jakou součást ostatních tříd, které ji
  referencují). Můžeme snadno použít guard kontrolující, že se nejedná
  o třídu s vlastností isEmbedded.}


Libovolnou množinu mapovacích pravidel nazýváme \emph{mapováním}
(obvykle dále značená jako $\lambda$). Množinu všech mapování
označujeme jako $\Lambda = \ps(\Omega))$. Mapování
sdružuje jednoduché mapovací funkce/pravidla a umožňuje popsat
složitější transformaci. \emph{Existence termínu mapování nám také v
  budoucnu usnadní mluvit např. o tom, jestli je sada mapovacích
  pravidel úplná, nekonfliktní atd. }

Mírným rozšířením definujeme obraz elementu $e \in E$ při mapování
$\lambda \in \Lambda$ jako
\begin{equation}
db(e, \lambda) : E \times
  \Lambda \to \ps(D) = \bigcup_{\rho \in \lambda}{db(e, \rho)}.
\end{equation}

Zcela analogicky definujme ještě obraz modelu $M \in \ps(E)$ při
mapování $\lambda \in \Lambda$ takto:
\begin{equation}
db(M, \lambda) : \ps(E)\times \Lambda \to \ps(D)
= \bigcup_{\omega \in \lambda, e \in E}{db(e, \omega)} = \bigcup_{e \in E}{db(e,
  \lambda)}.
\end{equation}

\emph{
\\TODO: Motivace obrazů, vzory, úplnost, ..
}

\section{Popis mapovacích pravidel}
\label{sec:rules}

\emph{
\\TODO: ...
}


%% \begin{figure}[hb]
%%   \begin{center}
%%     \includegraphics[width=.9\textwidth]{../graphics/lang-settings-locale-base}
%%   \end{center}
%%   \caption{Nastavení jazyka}
%%   \label{fig:lang-base-setting}
%% \end{figure}
